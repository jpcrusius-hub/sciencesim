\documentclass[11pt,a4paper]{article}
\usepackage[left=1.5cm,right=1.5cm,top=1.5cm,bottom=2cm,headheight=26pt]{geometry}
\usepackage[ngerman]{babel}
\usepackage[utf8]{inputenc}
\usepackage[T1]{fontenc}
\usepackage{amsmath,amssymb}
\usepackage{fancyhdr}
\usepackage{tcolorbox}
\usepackage{multicol}
\usepackage{qrcode}
\usepackage{tikz}
\usepackage{enumitem}
\usepackage{array}
\usepackage{booktabs}
\usepackage{xcolor}

\renewcommand{\familydefault}{\sfdefault}
\setlength{\parindent}{0pt}
\setlist{nosep,itemsep=2pt,topsep=3pt,partopsep=0pt}

\pagestyle{fancy}
\fancyhf{}
\lhead{\textbf{\textcolor{red}{MUSTERLÖSUNG}}}
\rhead{Stehende Wellen}
\cfoot{\scriptsize Seite \thepage}
\renewcommand{\headrulewidth}{0.4pt}

\begin{document}

\begin{center}
\Large\textbf{\textcolor{red}{MUSTERLÖSUNG} -- Stehende Wellen}\\[0.2cm]
\small Bearbeitungszeit: 35 Minuten \hfill Hilfsmittel: Simulation, Taschenrechner
\end{center}

\vspace{0.2cm}

% Bewertungsmatrix (15-NP für Sek II)
\begin{center}
\tiny
\begin{tabular}{|c|c|c|c|c|c|c|c|c|c|c|c|c|c|c|c|}
\hline
NP & 15 & 14 & 13 & 12 & 11 & 10 & 9 & 8 & 7 & 6 & 5 & 4 & 3 & 2 & 1 \\
\hline
\% & 95 & 90 & 85 & 80 & 75 & 70 & 65 & 60 & 55 & 50 & 45 & 40 & 33 & 27 & 20 \\
\hline
P & 28,5 & 27 & 25,5 & 24 & 22,5 & 21 & 19,5 & 18 & 16,5 & 15 & 13,5 & 12 & 10 & 8 & 6 \\
\hline
\end{tabular}
\end{center}

\begin{center}
\small Gesamtpunkte: \textbf{30 BE}
\end{center}

\vspace{0.3cm}

\begin{multicols}{2}
\setlength{\columnseprule}{0.4pt}

% === INFOBOX ===
\begin{tcolorbox}[colback=blue!5,colframe=blue!40!black,coltitle=black,colbacktitle=blue!10,
    title=\textbf{Simulation: Stehende Wellen}]
\small
Öffne die Simulation. Du siehst:
\begin{itemize}[leftmargin=*]
\item \textcolor{blue}{\textbf{Blaue Welle:}} läuft nach rechts
\item \textcolor{red}{\textbf{Rote Welle:}} läuft nach links (reflektiert)
\item \textcolor{green!50!black}{\textbf{Grüne Welle:}} Resultierende (stehende Welle)
\item \textcolor{yellow!80!black}{\textbf{Gelbe Punkte (K):}} Knoten
\item \textcolor{violet}{\textbf{Violette Linien (B):}} Bäuche
\end{itemize}

Du kannst $\lambda$ (2--8 cm), $A$ (0,5--2 cm) und $f$ (0,3--2 Hz) mit den Reglern ändern.

\vspace{0.2cm}
\begin{center}
\qrcode[height=2.2cm]{https://jpcrusius-hub.github.io/sciencesim/physik/wellen/stehende-wellen/}\\[0.2cm]
\tiny Simulation öffnen
\end{center}
\end{tcolorbox}

\vspace{0.3cm}

% === AUFGABE 1 ===
\begin{tcolorbox}[colback=white,colframe=black,coltitle=black,colbacktitle=white,
    title=\textbf{1: Beobachtung der Superposition}]

\textbf{a)} Stelle die Simulation auf Standardwerte ($\lambda = 4{,}0$ cm, $A = 1{,}0$ cm, $f = 1{,}0$ Hz). Beschreibe, was du beobachtest, wenn beide Einzelwellen angezeigt werden. \textbf{(3P)}

\textcolor{red}{Die blaue und rote Welle laufen in entgegengesetzte Richtungen {\color{blue}(1P)}. An manchen Stellen verstärken sie sich (konstruktive Interferenz), an anderen löschen sie sich aus (destruktive Interferenz) {\color{blue}(1P)}. Die grüne Resultierende schwankt auf und ab, bewegt sich aber nicht nach links oder rechts {\color{blue}(1P)}.}

\vspace{0.3cm}

\textbf{b)} Blende die Einzelwellen aus (nur grüne Welle sichtbar). Erkläre, warum diese Welle „stehend" genannt wird. \textbf{(2P)}

\textcolor{red}{Die Wellenberge und -täler bewegen sich nicht vorwärts {\color{blue}(1P)}. Die Welle schwingt nur auf und ab an festen Orten -- sie „steht" im Raum {\color{blue}(1P)}.}

\vspace{0.3cm}

\textbf{c)} Nenne die zwei Bedingungen, damit eine stabile stehende Welle entsteht (Kohärenzbedingung). \textbf{(2P)}

\textcolor{red}{1. Gleiche Frequenz der beiden Wellen {\color{blue}(1P)}\\
2. Konstante Phasendifferenz (feste Phasenbeziehung) {\color{blue}(1P)}}

\end{tcolorbox}

% === AUFGABE 2 ===
\begin{tcolorbox}[colback=white,colframe=black,coltitle=black,colbacktitle=white,
    title=\textbf{2: Knoten und Bäuche}]

\textbf{a)} Definiere die Begriffe \emph{Knoten} und \emph{Bauch} einer stehenden Welle. \textbf{(2P)}

Knoten: \textcolor{red}{Orte, an denen die Auslenkung dauerhaft null ist ($y = 0$) {\color{blue}(1P)}}

Bauch: \textcolor{red}{Orte mit maximaler Schwingungsamplitude ($y_{\max} = 2A$) {\color{blue}(1P)}}

\vspace{0.3cm}

\textbf{b)} Lies in der Simulation den Knotenabstand für $\lambda = 4{,}0$ cm ab. \textbf{(1P)}

Knotenabstand = \textcolor{red}{\underline{\hspace{0.5cm}2,0\hspace{0.5cm}}} cm {\color{blue}(1P)}

\vspace{0.3cm}

\textbf{c)} Ändere die Wellenlänge auf $\lambda = 6{,}0$ cm. Wie groß ist jetzt der Knotenabstand? \textbf{(1P)}

Knotenabstand = \textcolor{red}{\underline{\hspace{0.5cm}3,0\hspace{0.5cm}}} cm {\color{blue}(1P)}

\vspace{0.3cm}

\textbf{d)} Leite aus deinen Messungen die allgemeine Formel für den Knotenabstand her. \textbf{(2P)}

\textcolor{red}{$\lambda = 4{,}0$ cm $\rightarrow$ $\Delta x_K = 2{,}0$ cm $\rightarrow$ $\Delta x_K = \frac{4{,}0}{2} = 2{,}0$ {\color{blue}(1P)}\\
$\lambda = 6{,}0$ cm $\rightarrow$ $\Delta x_K = 3{,}0$ cm $\rightarrow$ $\Delta x_K = \frac{6{,}0}{2} = 3{,}0$\\[0.2cm]
Allgemein: $\boxed{\Delta x_K = \frac{\lambda}{2}}$ {\color{blue}(1P)}}

\end{tcolorbox}

\columnbreak

% === AUFGABE 3 ===
\begin{tcolorbox}[colback=white,colframe=black,coltitle=black,colbacktitle=white,
    title=\textbf{3: Wellenlängenbestimmung mit Mikrowellen}]

\textbf{a)} In einem Experiment mit Mikrowellen wird eine stehende Welle erzeugt. Der gemessene Abstand zwischen zwei benachbarten Knoten beträgt $\Delta x_K = 1{,}5\,\mathrm{cm}$. Berechne die Wellenlänge. \textbf{(3P)}

\textcolor{red}{Geg.: $\Delta x_K = 1{,}5\,\mathrm{cm}$ {\color{blue}(0,5P)}\\
Ges.: $\lambda$\\[0.2cm]
Lsg.: $\Delta x_K = \frac{\lambda}{2}$ {\color{blue}(1P)}\\[0.1cm]
$\lambda = 2 \cdot \Delta x_K = 2 \cdot 1{,}5\,\mathrm{cm}$ {\color{blue}(1P)}\\[0.1cm]
$\boxed{\lambda = 3{,}0\,\mathrm{cm}}$ {\color{blue}(0,5P)}}

\vspace{0.3cm}

\textbf{b)} Die Frequenz des Mikrowellensenders beträgt $f = 10\,\mathrm{GHz}$. Berechne die Ausbreitungsgeschwindigkeit der Mikrowellen. \textbf{(3P)}

\textcolor{red}{Geg.: $\lambda = 3{,}0\,\mathrm{cm} = 0{,}03\,\mathrm{m}$, $f = 10\,\mathrm{GHz} = 10 \cdot 10^9\,\mathrm{Hz}$ {\color{blue}(0,5P)}\\
Ges.: $v$\\[0.2cm]
Lsg.: $v = \lambda \cdot f$ {\color{blue}(1P)}\\[0.1cm]
$v = 0{,}03\,\mathrm{m} \cdot 10 \cdot 10^9\,\mathrm{Hz}$ {\color{blue}(1P)}\\[0.1cm]
$\boxed{v = 3 \cdot 10^8\,\mathrm{m/s}}$ {\color{blue}(0,5P)}}

\vspace{0.3cm}

\textbf{c)} Vergleiche dein Ergebnis mit der Lichtgeschwindigkeit $c = 3 \cdot 10^8\,\mathrm{m/s}$. Was fällt dir auf? Erkläre. \textbf{(2P)}

\textcolor{red}{Die berechnete Geschwindigkeit entspricht exakt der Lichtgeschwindigkeit {\color{blue}(1P)}. Mikrowellen sind elektromagnetische Wellen und breiten sich daher mit Lichtgeschwindigkeit aus {\color{blue}(1P)}.}

\end{tcolorbox}

% === AUFGABE 4 ===
\begin{tcolorbox}[colback=white,colframe=black,coltitle=black,colbacktitle=white,
    title=\textbf{4: Anwendung und Transfer}]

\textbf{a)} Zeichne eine stehende Welle mit genau 3 Knoten und 2 Bäuchen. Beschrifte Knoten (K), Bäuche (B) und den Knotenabstand $\frac{\lambda}{2}$. \textbf{(4P)}

\vspace{0.2cm}
\begin{center}
\begin{tikzpicture}[scale=0.9]
\draw[thick,->] (0,0) -- (8,0) node[right] {\scriptsize $x$};
\draw[thick,->] (0,-1.5) -- (0,1.5) node[above] {\scriptsize $y$};
% Stehende Welle
\draw[green!50!black,very thick,domain=0:7,samples=100] plot (\x,{1.2*sin(180*\x/3.5)});
% Knoten
\foreach \x in {0,3.5,7} {
    \fill[yellow!80!black] (\x,0) circle (4pt);
    \node[below,font=\scriptsize] at (\x,-0.3) {K};
}
% Bäuche
\foreach \x in {1.75,5.25} {
    \draw[violet,thick,dashed] (\x,-1.2) -- (\x,1.2);
    \node[above,font=\scriptsize,violet] at (\x,1.3) {B};
}
% Knotenabstand
\draw[<->,thick] (0,-1.1) -- (3.5,-1.1);
\node[below,font=\scriptsize] at (1.75,-1.1) {$\frac{\lambda}{2}$};
\end{tikzpicture}
\end{center}

\textcolor{red}{Punkte: Sinusform {\color{blue}(1P)}, 3 Knoten korrekt {\color{blue}(1P)}, 2 Bäuche korrekt {\color{blue}(1P)}, Beschriftung $\frac{\lambda}{2}$ {\color{blue}(1P)}}

\vspace{0.2cm}

\textbf{b)} Erkläre, warum es in einem Mikrowellenofen „heiße" und „kalte" Stellen gibt. Nutze dein Wissen über stehende Wellen. \textbf{(3P)}

\textcolor{red}{Im Mikrowellenofen bilden sich durch Reflexion an den Wänden stehende Wellen {\color{blue}(1P)}. An den Bäuchen ist die Energiedichte hoch -- dort wird das Essen stark erwärmt (,,heiße Stellen``) {\color{blue}(1P)}. An den Knoten ist die Energiedichte niedrig -- dort bleibt das Essen kalt (,,kalte Stellen``) {\color{blue}(1P)}.}

\vspace{0.2cm}

\textbf{c)} Der Drehteller im Mikrowellenofen hat einen Durchmesser von ca. 30 cm. Erkläre, warum er sich dreht. \textbf{(2P)}

\textcolor{red}{Der Drehteller bewegt das Essen durch Knoten und Bäuche {\color{blue}(1P)}, sodass alle Bereiche abwechselnd in Zonen hoher Energie gelangen und das Essen gleichmäßig erwärmt wird {\color{blue}(1P)}.}

\end{tcolorbox}

\end{multicols}

\vspace{0.5cm}
\begin{center}
\textbf{Punkteübersicht:} Aufgabe 1: 7P | Aufgabe 2: 6P | Aufgabe 3: 8P | Aufgabe 4: 9P | \textbf{Gesamt: 30 BE}
\end{center}

\end{document}
