\documentclass[11pt,a4paper]{article}
\usepackage[left=1.5cm,right=1.5cm,top=1.5cm,bottom=2cm,headheight=26pt]{geometry}
\usepackage[ngerman]{babel}
\usepackage[utf8]{inputenc}
\usepackage[T1]{fontenc}
\usepackage{amsmath,amssymb}
\usepackage{fancyhdr}
\usepackage{tcolorbox}
\usepackage{multicol}
\usepackage{qrcode}
\usepackage{tikz}
\usepackage{enumitem}
\usepackage{array}
\usepackage{booktabs}

\renewcommand{\familydefault}{\sfdefault}
\setlength{\parindent}{0pt}
\setlist{nosep,itemsep=2pt,topsep=3pt,partopsep=0pt}

\pagestyle{fancy}
\fancyhf{}
\lhead{Name: \underline{\hspace{5cm}}}
\rhead{Klasse: \underline{\hspace{1.5cm}} Datum: \underline{\hspace{2.5cm}}}
\cfoot{\scriptsize Seite \thepage}
\renewcommand{\headrulewidth}{0.4pt}

\begin{document}

\begin{center}
\Large\textbf{Stehende Wellen -- Arbeitsblatt mit Simulation}\\[0.2cm]
\small Bearbeitungszeit: 35 Minuten \hfill Hilfsmittel: Simulation, Taschenrechner
\end{center}

\vspace{0.2cm}

% Bewertungsmatrix (15-NP für Sek II)
\begin{center}
\tiny
\begin{tabular}{|c|c|c|c|c|c|c|c|c|c|c|c|c|c|c|c|}
\hline
NP & 15 & 14 & 13 & 12 & 11 & 10 & 9 & 8 & 7 & 6 & 5 & 4 & 3 & 2 & 1 \\
\hline
\% & 95 & 90 & 85 & 80 & 75 & 70 & 65 & 60 & 55 & 50 & 45 & 40 & 33 & 27 & 20 \\
\hline
P & 28,5 & 27 & 25,5 & 24 & 22,5 & 21 & 19,5 & 18 & 16,5 & 15 & 13,5 & 12 & 10 & 8 & 6 \\
\hline
\end{tabular}
\end{center}

\begin{center}
\small Gesamtpunkte: \textbf{30 BE}
\end{center}

\vspace{0.3cm}

\begin{multicols}{2}
\setlength{\columnseprule}{0.4pt}

% === INFOBOX ===
\begin{tcolorbox}[colback=blue!5,colframe=blue!40!black,coltitle=black,colbacktitle=blue!10,
    title=\textbf{Simulation: Stehende Wellen}]
\small
Öffne die Simulation. Du siehst:
\begin{itemize}[leftmargin=*]
\item \textcolor{blue}{\textbf{Blaue Welle:}} läuft nach rechts
\item \textcolor{red}{\textbf{Rote Welle:}} läuft nach links (reflektiert)
\item \textcolor{green!50!black}{\textbf{Grüne Welle:}} Resultierende (stehende Welle)
\item \textcolor{yellow!80!black}{\textbf{Gelbe Punkte (K):}} Knoten
\item \textcolor{violet}{\textbf{Violette Linien (B):}} Bäuche
\end{itemize}

Du kannst $\lambda$ (2--8 cm), $A$ (0,5--2 cm) und $f$ (0,3--2 Hz) mit den Reglern ändern.

\vspace{0.2cm}
\begin{center}
\qrcode[height=2.2cm]{https://jpcrusius-hub.github.io/sciencesim/physik/wellen/stehende-wellen/}\\[0.2cm]
\tiny Simulation öffnen
\end{center}
\end{tcolorbox}

\vspace{0.3cm}

% === AUFGABE 1 ===
\begin{tcolorbox}[colback=white,colframe=black,coltitle=black,colbacktitle=white,
    title=\textbf{1: Beobachtung der Superposition}]

\textbf{a)} Stelle die Simulation auf Standardwerte ($\lambda = 4{,}0$ cm, $A = 1{,}0$ cm, $f = 1{,}0$ Hz). Beschreibe, was du beobachtest, wenn beide Einzelwellen angezeigt werden. \textbf{(3P)}

\vspace{2.5cm}

\textbf{b)} Blende die Einzelwellen aus (nur grüne Welle sichtbar). Erkläre, warum diese Welle „stehend" genannt wird. \textbf{(2P)}

\vspace{2cm}

\textbf{c)} Nenne die zwei Bedingungen, damit eine stabile stehende Welle entsteht (Kohärenzbedingung). \textbf{(2P)}

\vspace{1.8cm}

\end{tcolorbox}

% === AUFGABE 2 ===
\begin{tcolorbox}[colback=white,colframe=black,coltitle=black,colbacktitle=white,
    title=\textbf{2: Knoten und Bäuche}]

\textbf{a)} Definiere die Begriffe \emph{Knoten} und \emph{Bauch} einer stehenden Welle. \textbf{(2P)}

\vspace{0.2cm}
Knoten: \hrulefill

\vspace{0.8cm}

Bauch: \hrulefill

\vspace{0.8cm}

\textbf{b)} Lies in der Simulation den Knotenabstand für $\lambda = 4{,}0$ cm ab. \textbf{(1P)}

\vspace{0.3cm}
Knotenabstand = \underline{\hspace{2cm}} cm

\vspace{0.3cm}

\textbf{c)} Ändere die Wellenlänge auf $\lambda = 6{,}0$ cm. Wie groß ist jetzt der Knotenabstand? \textbf{(1P)}

\vspace{0.3cm}
Knotenabstand = \underline{\hspace{2cm}} cm

\vspace{0.3cm}

\textbf{d)} Leite aus deinen Messungen die allgemeine Formel für den Knotenabstand her. \textbf{(2P)}

\vspace{1.5cm}

\end{tcolorbox}

\columnbreak

% === AUFGABE 3 ===
\begin{tcolorbox}[colback=white,colframe=black,coltitle=black,colbacktitle=white,
    title=\textbf{3: Wellenlängenbestimmung mit Mikrowellen}]

\textbf{a)} In einem Experiment mit Mikrowellen wird eine stehende Welle erzeugt. Der gemessene Abstand zwischen zwei benachbarten Knoten beträgt $\Delta x_K = 1{,}5\,\mathrm{cm}$. Berechne die Wellenlänge. \textbf{(3P)}

\vspace{2.8cm}

\textbf{b)} Die Frequenz des Mikrowellensenders beträgt $f = 10\,\mathrm{GHz}$. Berechne die Ausbreitungsgeschwindigkeit der Mikrowellen. \textbf{(3P)}

\vspace{2.8cm}

\textbf{c)} Vergleiche dein Ergebnis mit der Lichtgeschwindigkeit $c = 3 \cdot 10^8\,\mathrm{m/s}$. Was fällt dir auf? Erkläre. \textbf{(2P)}

\vspace{2cm}

\end{tcolorbox}

% === AUFGABE 4 ===
\begin{tcolorbox}[colback=white,colframe=black,coltitle=black,colbacktitle=white,
    title=\textbf{4: Anwendung und Transfer}]

\textbf{a)} Zeichne eine stehende Welle mit genau 3 Knoten und 2 Bäuchen. Beschrifte Knoten (K), Bäuche (B) und den Knotenabstand $\frac{\lambda}{2}$. \textbf{(4P)}

\vspace{0.3cm}
\begin{center}
\begin{tikzpicture}
\draw[thick,->] (0,0) -- (8,0) node[right] {\scriptsize $x$};
\draw[thick,->] (0,-1.5) -- (0,1.5) node[above] {\scriptsize $y$};
\draw[gray,very thin] (0,-1.5) grid[step=0.5] (7.5,1.5);
\end{tikzpicture}
\end{center}

\vspace{0.3cm}

\textbf{b)} Erkläre, warum es in einem Mikrowellenofen „heiße" und „kalte" Stellen gibt. Nutze dein Wissen über stehende Wellen. \textbf{(3P)}

\vspace{2.5cm}

\textbf{c)} Der Drehteller im Mikrowellenofen hat einen Durchmesser von ca. 30 cm. Erkläre, warum er sich dreht. \textbf{(2P)}

\vspace{2cm}

\end{tcolorbox}

\end{multicols}

\end{document}
