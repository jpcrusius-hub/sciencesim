\documentclass[11pt,a4paper,landscape]{article}
\usepackage[utf8]{inputenc}
\usepackage[T1]{fontenc}
\usepackage[ngerman]{babel}
\usepackage[left=1.5cm,right=1.5cm,top=1.5cm,bottom=1.5cm]{geometry}
\usepackage{amsmath,amssymb}
\usepackage{multicol}
\usepackage{tikz}
\usepackage{array}
\usepackage{xcolor}

% Serifenlose Schrift
\renewcommand{\familydefault}{\sfdefault}

\pagestyle{empty}

\begin{document}

\begin{center}
\LARGE\textbf{Bewegungsarten}
\end{center}

\vspace{5mm}

\begin{multicols}{3}

% Spalte 1: Bahnform
\section*{Bahnform}
\textit{Welchen Weg nimmt das Objekt?}

\vspace{3mm}

\subsection*{1. Geradlinig}
\begin{center}
\begin{tikzpicture}[scale=0.8]
    \draw[thick,->] (0,0) -- (4,0);
    \fill[blue] (0.5,0) circle (0.15);
    \fill[blue] (2,0) circle (0.15);
    \fill[blue] (3.5,0) circle (0.15);
    \node[below] at (2,-0.3) {\small gerade Linie};
\end{tikzpicture}
\end{center}
\textbf{Beispiele:} Auto, Rolltreppe, Fahrstuhl

\vspace{4mm}

\subsection*{2. Kreisförmig}
\begin{center}
\begin{tikzpicture}[scale=0.8]
    \draw[thick,->] (1.5,0) arc (0:330:1.5);
    \fill[blue] (1.5,0) circle (0.15);
    \fill[blue] (0,1.5) circle (0.15);
    \fill[blue] (-1.5,0) circle (0.15);
    \node[below] at (0,-1.8) {\small Kreisbahn};
\end{tikzpicture}
\end{center}
\textbf{Beispiele:} Karussell, Uhrzeiger, Riesenrad

\vspace{4mm}

\subsection*{3. Krummlinig}
\begin{center}
\begin{tikzpicture}[scale=0.8]
    \draw[thick,->] (0,0) .. controls (1,1.5) and (2,-0.5) .. (3.5,1);
    \fill[blue] (0,0) circle (0.15);
    \fill[blue] (1.2,1.1) circle (0.15);
    \fill[blue] (2.5,0.2) circle (0.15);
    \node[below] at (1.75,-0.5) {\small beliebige Kurve};
\end{tikzpicture}
\end{center}
\textbf{Beispiele:} Achterbahn, Schaukel

\columnbreak

% Spalte 2: Bewegungsart
\section*{Bewegungsart}
\textit{Wie ändert sich die Geschwindigkeit?}

\vspace{3mm}

\subsection*{1. Gleichförmig}
\begin{center}
\begin{tikzpicture}[scale=0.8]
    \draw[thick,->] (0,0) -- (4,0);
    \fill[blue] (1,0) circle (0.15);
    \draw[thick,->,red] (1.2,0.3) -- (2.2,0.3);
    \node[above,red] at (1.7,0.4) {\small $v$};
    \fill[blue] (3,0) circle (0.15);
    \draw[thick,->,red] (3.2,0.3) -- (4.2,0.3);
    \node[above,red] at (3.7,0.4) {\small $v$};
\end{tikzpicture}
\end{center}

\begin{center}
\fbox{\parbox{5cm}{\centering\textbf{Geschwindigkeit bleibt gleich}\\[2mm] $v = \text{konstant}$}}
\end{center}

\textbf{Beispiele:} Auto auf Autobahn, Rolltreppe, Uhrzeiger

\vspace{6mm}

\subsection*{2. Ungleichförmig}
\begin{center}
\begin{tikzpicture}[scale=0.8]
    \draw[thick,->] (0,0) -- (4,0);
    \fill[blue] (0.5,0) circle (0.15);
    \draw[thick,->,red] (0.7,0.3) -- (1.2,0.3);
    \node[above,red] at (0.95,0.4) {\scriptsize $v$};
    \fill[blue] (2,0) circle (0.15);
    \draw[thick,->,red] (2.2,0.3) -- (3.2,0.3);
    \node[above,red] at (2.7,0.4) {\scriptsize $v$};
    \fill[blue] (3.5,0) circle (0.15);
    \draw[thick,->,red] (3.7,0.3) -- (5.2,0.3);
    \node[above,red] at (4.5,0.4) {\scriptsize $v$};
\end{tikzpicture}
\end{center}

\begin{center}
\fbox{\parbox{5cm}{\centering\textbf{Geschwindigkeit ändert sich}\\[2mm] schneller $\leftrightarrow$ langsamer}}
\end{center}

\textbf{Beispiele:} Anfahren, Bremsen, Ball werfen, Schaukel

\columnbreak

% Spalte 3: Zusammenfassung
\section*{Übersicht}

\vspace{3mm}

\renewcommand{\arraystretch}{1.4}
\begin{tabular}{|p{2.2cm}|p{2.2cm}|p{2.2cm}|}
\hline
 & \textbf{gleichf.} & \textbf{ungleichf.} \\
\hline
\textbf{geradlinig} & Auto, Rolltreppe & Anfahren, Ball \\
\hline
\textbf{kreisförmig} & Karussell, Uhr & Karussell start \\
\hline
\textbf{krummlinig} & (selten) & Achterbahn, Schaukel \\
\hline
\end{tabular}

\vspace{8mm}

\section*{Merksätze}

\vspace{2mm}

\fbox{\parbox{6.5cm}{
\textbf{1.} Die \textbf{Bahnform} beschreibt den \textbf{Weg}.
}}

\vspace{4mm}

\fbox{\parbox{6.5cm}{
\textbf{2.} Die \textbf{Bewegungsart} beschreibt die \textbf{Geschwindigkeit}.
}}

\vspace{4mm}

\fbox{\parbox{6.5cm}{
\textbf{3.} Jede Bewegung hat \textbf{beides}: eine Bahnform UND eine Bewegungsart.
}}

\vspace{8mm}

\begin{center}
\textit{Datum: \underline{\hspace{3cm}}}
\end{center}

\end{multicols}

\end{document}
