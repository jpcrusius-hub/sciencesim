\documentclass[11pt,a4paper]{article}
\usepackage[utf8]{inputenc}
\usepackage[T1]{fontenc}
\usepackage[ngerman]{babel}
\usepackage[left=1.5cm,right=1.5cm,top=1.5cm,bottom=2cm,headheight=26pt]{geometry}
\usepackage{amsmath,amssymb}
\usepackage{array,booktabs,tabularx}
\usepackage{multicol}
\usepackage{tcolorbox}
\usepackage{fancyhdr}
\usepackage{enumitem}
\usepackage{tikz}
\usepackage{xcolor}
\usepackage{qrcode}

% Serifenlose Schrift
\renewcommand{\familydefault}{\sfdefault}

% Kompakte Listen
\setlist{nosep,itemsep=2pt,topsep=3pt,partopsep=0pt}

% Kopf- und Fußzeile
\pagestyle{fancy}
\fancyhf{}
\lhead{Name: \underline{\hspace{5cm}}}
\rhead{Klasse: \underline{\hspace{1.5cm}} Datum: \underline{\hspace{2.5cm}}}
\cfoot{\scriptsize Seite \thepage}
\renewcommand{\headrulewidth}{0.4pt}

% tcolorbox Stil
\tcbset{
    colback=white,
    colframe=black,
    coltitle=black,
    colbacktitle=white,
    boxrule=0.5pt,
    arc=0pt,
    left=3pt,
    right=3pt,
    top=3pt,
    bottom=3pt,
    fonttitle=\bfseries
}

\begin{document}

% Titel
\begin{center}
\Large\textbf{Bewegungsarten erforschen}\\[3pt]
\small Bearbeitungszeit: 45 Minuten \hfill Hilfsmittel: keine
\end{center}

% Bewertungsmatrix
\begin{center}
\tiny
\setlength{\tabcolsep}{2pt}
\begin{tabular}{|l|c|c|c|c|c|c|c|c|c|c|c|c|c|c|}
\hline
\textbf{NP} & 14 & 13 & 12 & 11 & 10 & 9 & 8 & 7 & 6 & 5 & 4 & 3 & 2 & 1 \\
\hline
\textbf{\%} & 100 & 96 & 90,7 & 86 & 80 & 73,3 & 66,7 & 60 & 53,3 & 46,7 & 40 & 33,3 & 26,7 & 20 \\
\hline
\textbf{P(*)} & 26 & 25 & 23,5 & 22,5 & 21 & 19 & 17,5 & 15,5 & 14 & 12 & 10,5 & 8,5 & 7 & 5 \\
\hline
\textbf{P(**)} & 31 & 29,5 & 28 & 26,5 & 25 & 22,5 & 20,5 & 18,5 & 16,5 & 14,5 & 12,5 & 10,5 & 8,5 & 6 \\
\hline
\textbf{P(***)} & 35 & 33,5 & 31,5 & 30 & 28 & 25,5 & 23,5 & 21 & 18,5 & 16,5 & 14 & 11,5 & 9,5 & 7 \\
\hline
\end{tabular}
\end{center}

\begin{center}
\small Gesamtpunkte: \textbf{(*): 26 / (**): 31 / (***): 35}
\end{center}

% Simulations-Hinweis
\begin{center}
\small
\fbox{\parbox{0.85\textwidth}{
\begin{minipage}{0.12\textwidth}
\centering
\qrcode[height=1.5cm]{https://jpcrusius-hub.github.io/sciencesim/physik/mechanik/bewegungsarten/}
\end{minipage}
\hfill
\begin{minipage}{0.83\textwidth}
\textbf{Simulation:} Öffne die Simulation auf deinem iPad oder scanne den QR-Code.\\
Beobachte alle 8 Bewegungen und achte auf die \textbf{Bahn} (gestrichelte Linie) und den \textbf{Geschwindigkeitspfeil}.
\end{minipage}
}}
\end{center}

\vspace{3mm}

\setlength{\columnseprule}{0.4pt}
\begin{multicols}{2}

% Aufgabe 1: Klassifizierung
\begin{tcolorbox}[title={1: Klassifiziere die Bewegungen}]
\small
(*) Beobachte die Bewegungen in der Simulation. Kreuze die richtige Bahnform und Bewegungsart an. \textbf{(8P)}

\vspace{2mm}
\scriptsize
\setlength{\tabcolsep}{3pt}
\begin{tabular}{|c|p{2.4cm}|c|c|c|c|c|}
\hline
\textbf{\#} & \textbf{Bewegung} & \rotatebox{90}{\textbf{geradlinig}~} & \rotatebox{90}{\textbf{kreisförm.}~} & \rotatebox{90}{\textbf{krummlin.}~} & \rotatebox{90}{\textbf{gleichförm.}~} & \rotatebox{90}{\textbf{ungleichf.}~} \\
\hline
1 & Auto Autobahn & $\square$ & $\square$ & $\square$ & $\square$ & $\square$ \\
\hline
2 & Ball hochwerfen & $\square$ & $\square$ & $\square$ & $\square$ & $\square$ \\
\hline
3 & Karussell & $\square$ & $\square$ & $\square$ & $\square$ & $\square$ \\
\hline
4 & Fahrrad anfahren & $\square$ & $\square$ & $\square$ & $\square$ & $\square$ \\
\hline
5 & Uhrzeiger & $\square$ & $\square$ & $\square$ & $\square$ & $\square$ \\
\hline
6 & Achterbahn & $\square$ & $\square$ & $\square$ & $\square$ & $\square$ \\
\hline
7 & Schaukel & $\square$ & $\square$ & $\square$ & $\square$ & $\square$ \\
\hline
8 & Rolltreppe & $\square$ & $\square$ & $\square$ & $\square$ & $\square$ \\
\hline
\end{tabular}
\end{tcolorbox}

% Aufgabe 2: Lückentext
\begin{tcolorbox}[title={2: Lückentext Grundbegriffe}]
\small
(*) Ergänze die fehlenden Wörter. \textbf{(6P)}

\vspace{1mm}
\scriptsize
\begin{center}
\fbox{\parbox{6.5cm}{\centering geradlinig ~ kreisförmig ~ krummlinig\\[1pt] gleichförmig ~ ungleichförmig ~ Geschwindigkeit}}
\end{center}

\small
\vspace{2mm}
a) Bei der \textbf{Bahnform} unterscheidet man drei Arten: 

\underline{\hspace{2.5cm}}, \underline{\hspace{2.5cm}} und \underline{\hspace{2.5cm}}.

\vspace{3mm}
b) Bei einer \underline{\hspace{2.8cm}} Bewegung bleibt die \underline{\hspace{2.8cm}} gleich.

\vspace{3mm}
c) Bei einer \underline{\hspace{2.8cm}} Bewegung ändert sich die Geschwindigkeit.
\end{tcolorbox}

% Aufgabe 3: Definitionen
\begin{tcolorbox}[title={3: Erkläre die Begriffe}]
\small
\textbf{a) (*)} Erkläre: Was ist eine \textbf{gleichförmige Bewegung}? \textbf{(2P)}

\vspace{8mm}
\hrule
\vspace{6mm}
\hrule

\vspace{4mm}
\textbf{b) (*)} Erkläre: Was ist eine \textbf{ungleichförmige Bewegung}? \textbf{(2P)}

\vspace{8mm}
\hrule
\vspace{6mm}
\hrule
\end{tcolorbox}

\columnbreak

% Aufgabe 4: Richtig oder Falsch
\begin{tcolorbox}[title={4: Richtig oder Falsch?}]
\small
(*) Kreuze an: Richtig (R) oder Falsch (F)? \textbf{(5P)}

\vspace{2mm}
\begin{tabular}{|p{6.2cm}|c|c|}
\hline
\textbf{Aussage} & \textbf{R} & \textbf{F} \\
\hline
a) Ein Auto auf der Autobahn bewegt sich geradlinig. & $\square$ & $\square$ \\
\hline
b) Der Sekundenzeiger einer Uhr bewegt sich ungleichförmig. & $\square$ & $\square$ \\
\hline
c) Bei einer gleichförmigen Bewegung bleibt die Geschwindigkeit gleich. & $\square$ & $\square$ \\
\hline
d) Eine Schaukel bewegt sich auf einer Kreisbahn. & $\square$ & $\square$ \\
\hline
e) Beim Anfahren eines Fahrrads ist die Bewegung ungleichförmig. & $\square$ & $\square$ \\
\hline
\end{tabular}
\end{tcolorbox}

% Aufgabe 5: Eigene Beispiele
\begin{tcolorbox}[title={5: Eigene Beispiele aus dem Alltag}]
\small
Nenne jeweils ein eigenes Beispiel:

\vspace{2mm}
\textbf{a) (*)} Geradlinig und gleichförmig: \textbf{(1P)}

\vspace{5mm}
\hrule

\vspace{3mm}
\textbf{b) (*)} Geradlinig und ungleichförmig: \textbf{(1P)}

\vspace{5mm}
\hrule

\vspace{3mm}
\textbf{c) (*)} Kreisförmig und gleichförmig: \textbf{(1P)}

\vspace{5mm}
\hrule

\vspace{3mm}
\textbf{d) (**)} Kreisförmig und ungleichförmig: \textbf{(1P)}

\vspace{5mm}
\hrule

\vspace{3mm}
\textbf{e) (**)} Krummlinig und ungleichförmig: \textbf{(1P)}

\vspace{5mm}
\hrule

\vspace{3mm}
\textbf{f) (***)} Krummlinig und gleichförmig: \textbf{(1P)}

\vspace{5mm}
\hrule
\end{tcolorbox}

% Aufgabe 6: Vertiefung
\begin{tcolorbox}[title={6: Vertiefung Schaukel}]
\small
\textbf{(**)} Die Schaukel bewegt sich auf einem Bogen hin und her. Warum ist diese Bewegung \textbf{ungleichförmig}? 

\textit{Tipp: Beobachte den Geschwindigkeitspfeil!} \textbf{(3P)}

\vspace{8mm}
\hrule
\vspace{6mm}
\hrule
\vspace{6mm}
\hrule
\end{tcolorbox}

% Aufgabe 7: Transfer
\begin{tcolorbox}[title={7: Transferaufgabe}]
\small
\textbf{(***)} Ein Karussell ist eine \textbf{kreisförmige} Bewegung. Warum nennt man sie nicht einfach "`krummlinig"'? 

Was ist der Unterschied zwischen kreisförmig und krummlinig? \textbf{(3P)}

\vspace{8mm}
\hrule
\vspace{6mm}
\hrule
\vspace{6mm}
\hrule
\end{tcolorbox}

\end{multicols}

\end{document}
