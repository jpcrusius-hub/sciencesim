\documentclass[11pt,a4paper,landscape]{article}
\usepackage[utf8]{inputenc}
\usepackage[T1]{fontenc}
\usepackage[ngerman]{babel}
\usepackage[left=1.5cm,right=1.5cm,top=1.5cm,bottom=1.5cm]{geometry}
\usepackage{amsmath}
\usepackage{tikz}
\usetikzlibrary{arrows.meta,positioning,decorations.pathmorphing}
\usepackage{multicol}
\usepackage{tcolorbox}
\usepackage{enumitem}

\renewcommand{\familydefault}{\sfdefault}
\setlist{nosep,itemsep=2pt,topsep=3pt,partopsep=0pt}

\pagestyle{empty}

\begin{document}

\begin{center}
\LARGE\textbf{Volumenbestimmung unregelm\"a\ss iger K\"orper}
\end{center}

\vspace{0.3cm}

\begin{multicols}{3}

\subsection*{Problem}

Wie bestimmt man das Volumen eines Steins?

\vspace{0.2cm}
\begin{center}
\begin{tikzpicture}[scale=0.8]
\draw[thick,fill=gray!30] (0,0) to[out=30,in=150] (1.5,0.2) to[out=-30,in=60] (1.8,-0.5) to[out=240,in=-30] (0.3,-0.8) to[out=150,in=-120] (0,0);
\node at (0.9,-0.2) {?};
\draw[thick] (2.5,0) -- (4,0);
\foreach \x in {2.6,2.8,...,3.9}
  \draw (\x,0) -- (\x,0.15);
\draw[red,thick] (2.3,-0.3) -- (4.2,0.3);
\draw[red,thick] (2.3,0.3) -- (4.2,-0.3);
\end{tikzpicture}
\end{center}

\vspace{0.1cm}
$\rightarrow$ Unregelm\"a\ss ige Form!\\
$\rightarrow$ Lineal funktioniert nicht!

\vspace{0.3cm}
\hrule
\vspace{0.3cm}

\subsection*{1. Differenzmethode}

\begin{center}
\begin{tikzpicture}[scale=0.65]
% Messzylinder vorher
\begin{scope}[shift={(0,0)}]
  \draw[thick] (0,0) -- (0,3.8) -- (1.2,3.8) -- (1.2,0);
  \draw[thick] (0,0) arc (180:360:0.6 and 0.15);
  % Skala
  \foreach \y/\label in {0.7/20, 1.4/40, 2.1/60, 2.8/80}
    \draw (1.2,\y) -- (1.4,\y) node[right,font=\tiny] {\label};
  % Wasser mit Meniskus (80-90% horizontal, nur R\"ander hoch)
  \fill[blue!25] (0.05,0.05) -- (0.05,1.35) to[out=-45,in=180] (0.15,1.25) -- (1.05,1.25) to[out=0,in=-135] (1.15,1.35) -- (1.15,0.05) -- cycle;
  \draw[blue,thick] (0.05,1.35) to[out=-45,in=180] (0.15,1.25) -- (1.05,1.25) to[out=0,in=-135] (1.15,1.35);
  % Beschriftung
  \draw[<-,thick] (1.5,1.3) -- (2,1.3) node[right,font=\small] {$V_1$};
  \node[below,font=\small] at (0.6,-0.4) {vorher};
\end{scope}

% Pfeil
\draw[->,very thick] (2.8,1.5) -- (3.5,1.5);

% Messzylinder nachher
\begin{scope}[shift={(4,0)}]
  \draw[thick] (0,0) -- (0,3.8) -- (1.2,3.8) -- (1.2,0);
  \draw[thick] (0,0) arc (180:360:0.6 and 0.15);
  % Skala
  \foreach \y/\label in {0.7/20, 1.4/40, 2.1/60, 2.8/80}
    \draw (1.2,\y) -- (1.4,\y) node[right,font=\tiny] {\label};
  % Wasser mit Meniskus (80-90% horizontal, nur R\"ander hoch)
  \fill[blue!25] (0.05,0.05) -- (0.05,2.05) to[out=-45,in=180] (0.15,1.95) -- (1.05,1.95) to[out=0,in=-135] (1.15,2.05) -- (1.15,0.05) -- cycle;
  \draw[blue,thick] (0.05,2.05) to[out=-45,in=180] (0.15,1.95) -- (1.05,1.95) to[out=0,in=-135] (1.15,2.05);
  % K\"orper am Boden
  \fill[gray!50] (0.25,0.15) rectangle (0.95,0.75);
  % Beschriftung
  \draw[<-,thick] (1.5,2.0) -- (2,2.0) node[right,font=\small] {$V_2$};
  \node[below,font=\small] at (0.6,-0.4) {nachher};
\end{scope}
\end{tikzpicture}
\end{center}

\vspace{0.2cm}

\textbf{Durchf\"uhrung:}
\begin{enumerate}[label=\arabic*.]
\item Wasser einf\"ullen
\item $V_1$ ablesen
\item K\"orper eintauchen
\item $V_2$ ablesen
\item Berechnen
\end{enumerate}

\vspace{0.3cm}

\begin{tcolorbox}[colback=white,colframe=black,boxrule=1.5pt]
\centering
\large $V_{\textnormal{K\"orper}} = V_2 - V_1$
\end{tcolorbox}

\columnbreak

\subsection*{2. \"Uberlaufmethode}

\begin{center}
\begin{tikzpicture}[scale=0.8]
% \"Uberlaufgef\"a\ss (Becherglas mit schr\"ager T\"ulle)
\begin{scope}[shift={(0,0)}]
  % Becher (zylindrisch)
  \draw[thick] (0,0) -- (0,2.6) -- (1.8,2.6) -- (1.8,0);
  \draw[thick] (0,0) arc (180:360:0.9 and 0.22);
  
  % Wasser im Becher mit korrektem Meniskus (80-90% horizontal, nur R\"ander hoch)
  \fill[blue!25] (0.04,0.06) -- (0.04,2.0) to[out=-45,in=180] (0.15,1.9) -- (1.65,1.9) to[out=0,in=-135] (1.76,2.0) -- (1.76,0.06) arc (360:180:0.86 and 0.19);
  \draw[blue,thick] (0.04,2.0) to[out=-45,in=180] (0.15,1.9) -- (1.65,1.9) to[out=0,in=-135] (1.76,2.0);
  
  % Wasser in der T\"ulle (gef\"ulltes Rohr)
  \fill[blue!25] (1.8,2.1) -- (2.8,1.0) -- (2.8,0.8) -- (1.8,1.9) -- cycle;
  
  % T\"ulle Umriss (\"uber dem Wasser zeichnen)
  \draw[thick] (1.8,2.1) -- (2.8,1.0);  % Oberkante T\"ulle
  \draw[thick] (1.8,1.9) -- (2.8,0.8);  % Unterkante T\"ulle
  \draw[thick] (2.8,1.0) -- (2.8,0.8);  % T\"ullenende
  
  % K\"orper im Wasser (am Boden)
  \fill[gray!50] (0.4,0.24) rectangle (1.4,0.94);
  
  % Wasserstrahl (f\"allt vom T\"ullenende senkrecht)
  \draw[blue,thick,line width=1.5pt] (2.8,0.9) -- (2.8,-0.9);
  
  \node[below,font=\scriptsize] at (0.9,-0.45) {\"Uberlaufgef\"a\ss};
\end{scope}

% Messzylinder direkt unter der T\"ulle
\begin{scope}[shift={(2.35,-1.6)}]
  \draw[thick] (0,0) -- (0,1.6) -- (0.85,1.6) -- (0.85,0);
  \draw[thick] (0,0) arc (180:360:0.425 and 0.1);
  % Skala
  \foreach \y/\label in {0.35/10, 0.7/20, 1.05/30}
    \draw (0.85,\y) -- (1.0,\y) node[right,font=\tiny] {\label};
  % Aufgefangenes Wasser mit korrektem Meniskus (80-90% horizontal, nur R\"ander hoch)
  \fill[blue!25] (0.03,0.03) -- (0.03,0.65) to[out=-45,in=180] (0.1,0.55) -- (0.75,0.55) to[out=0,in=-135] (0.82,0.65) -- (0.82,0.03) -- cycle;
  \draw[blue,thick] (0.03,0.65) to[out=-45,in=180] (0.1,0.55) -- (0.75,0.55) to[out=0,in=-135] (0.82,0.65);
  \node[below,font=\scriptsize] at (0.425,-0.18) {Messzylinder};
\end{scope}
\end{tikzpicture}
\end{center}

\vspace{0.2cm}

\textbf{Durchf\"uhrung:}
\begin{enumerate}[label=\arabic*.]
\item Gef\"a\ss{} randvoll f\"ullen
\item Messzylinder unter Auslauf
\item K\"orper eintauchen
\item \"Ubergelaufenes Wasser ablesen
\end{enumerate}

\vspace{0.3cm}

\begin{tcolorbox}[colback=white,colframe=black,boxrule=1.5pt]
\centering
\large $V_{\textnormal{K\"orper}} = V_{\textnormal{aufgefangen}}$
\end{tcolorbox}

\vspace{0.3cm}
\hrule
\vspace{0.3cm}

\subsection*{Vergleich}

\begin{tabular}{|p{2.2cm}|p{2.2cm}|}
\hline
\textbf{Differenz} & \textbf{\"Uberlauf} \\
\hline
1 Gef\"a\ss & 2 Gef\"a\ss e \\
\hline
Berechnung n\"otig & Direkt ablesen \\
\hline
F\"ur kleine K\"orper & F\"ur gro\ss e K\"orper \\
\hline
\end{tabular}

\columnbreak

\subsection*{Warum funktioniert das?}

\begin{center}
\begin{tikzpicture}[scale=0.8]
\draw[thick,fill=gray!30] (0,0) rectangle (1.5,1.5);
\node at (0.75,0.75) {K\"orper};
\draw[->,very thick] (2,0.75) -- (3,0.75);
\draw[thick,fill=blue!25] (3.5,0) rectangle (5,1.5);
\node at (4.25,0.75) {Wasser};
\node at (2.5,0.75) {\Large =};
\end{tikzpicture}
\end{center}

Der K\"orper \textbf{verdr\"angt} genau so viel Wasser, wie er selbst Platz braucht.

\vspace{0.3cm}

\begin{tcolorbox}[colback=yellow!20,colframe=black,boxrule=2pt]
\centering
\textbf{Merksatz:}\\[0.2cm]
\large $V_{\textnormal{K\"orper}} = V_{\textnormal{verdr\"angtes Wasser}}$
\end{tcolorbox}

\vspace{0.4cm}
\hrule
\vspace{0.3cm}

\subsection*{Richtiges Ablesen}

\begin{center}
\begin{tikzpicture}[scale=0.85]
% Messzylinder
\draw[thick] (0,0) -- (0,3.2) -- (1.2,3.2) -- (1.2,0);
\draw[thick] (0,0) arc (180:360:0.6 and 0.15);
% Skala
\foreach \y/\label in {0.6/20, 1.2/40, 1.8/60, 2.4/80}
  \draw (1.2,\y) -- (1.4,\y) node[right,font=\tiny] {\label};
% Wasser mit korrektem Meniskus (80-90% horizontal, nur R\"ander hoch)
\fill[blue!25] (0.03,0.03) -- (0.03,1.55) to[out=-45,in=180] (0.12,1.4) -- (1.08,1.4) to[out=0,in=-135] (1.17,1.55) -- (1.17,0.03) -- cycle;
\draw[blue,thick] (0.03,1.55) to[out=-45,in=180] (0.12,1.4) -- (1.08,1.4) to[out=0,in=-135] (1.17,1.55);
% Ablese-Linie (am tiefsten Punkt)
\draw[red,dashed,thick] (-0.3,1.4) -- (1.8,1.4);
% Auge auf Höhe des Meniskus
\begin{scope}[shift={(-1.0,1.4)}]
  \draw[thick] (-0.25,0) -- (0,0.15) -- (0.25,0) -- (0,-0.15) -- cycle;
  \fill (0,0) circle (0.06);
\end{scope}
% Pfeil vom Auge
\draw[->,thick] (-0.7,1.4) -- (-0.35,1.4);
% Beschriftung
\node[right,font=\small] at (1.9,1.4) {hier ablesen!};
\draw[->] (1.85,1.4) -- (1.45,1.4);
\end{tikzpicture}
\end{center}

\vspace{0.2cm}

\begin{enumerate}[label=\arabic*.]
\item \textbf{Unten} am Meniskus ablesen
\item \textbf{Waagerecht} auf Augenh\"ohe
\end{enumerate}

\vspace{0.3cm}

\begin{tcolorbox}[colback=white,colframe=black]
\centering
\textbf{Einheiten:}\\[0.1cm]
$1\,\mathrm{ml} = 1\,\mathrm{cm}^3$
\end{tcolorbox}

\end{multicols}

\end{document}
